\chapter{Vorgehen}

Zur Erstellung unseres Modells für die Sentiment- und Intent-Klassifikation haben wir uns an 
einem etablierten Workflow aus dem Bereich der Textklassifikation orientiert. Der Workflow 
beschreibt die Schritte, die zur Entwicklung eines Textklassifikationsmodells durchlaufen werden 
können. In dem Workflow werden die folgenden sechs Schritte beschrieben:  Datenbeschaffung, Datenanalyse 
und -labeling, Feature-Konstruktion und -gewichtung, Feature-Selektion und -Projektion,  Modelltraining 
sowie Evaluation der Lösung. \cite{Mirończuk}

Im folgenden wird nur kurz auf die Schritte des Workflows eingegangen, die für 
unseren Anwendungsfall relevant waren. Datenbeschaffung die Daten wurden uns in Kundenmails in anonymisierter 
Form von Allbranded zur Verfügung gestellt. Im darauffolgenden Datenanalyse und-labelling Schritt wurden Labels 
für die Kundenmails erarbeitet und auf die Mails übertragen. Nach dem Labelling wurde direkt mit dem fünften 
Schritt dem Modelltraining begonnen. Zum abschluss eines Modelltrainings wurden diese dann anhand gäniger
Metriken evaluiert und mit den anderen Modellen verglichen. \cite{Mirończuk}


% \begin{figure}[H]
% \centering
% \includegraphics{./bilder/Workflow.png}
% \caption{Korrelation zwischen der abhängigen und der unabhängigen Variablen}
% \label{fig:Workflow}
% \end{figure}