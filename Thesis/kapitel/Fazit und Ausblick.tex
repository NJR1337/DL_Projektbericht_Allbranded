\chapter{Fazit und Ausblick}

Der Einsatz der Modelle über die entwickelte Schnittstelle erweist sich als praktikabel. Das Sentiment
wird mit dem trainierten Modell zuverlässig erkannt, während Intents, je nach Klassencharakteristik,
unterschiedlich gut identifiziert werden. Insbesondere häufig vorkommende Anfragetypen können effektiv
klassifiziert werden, wodurch eine Zeitersparnis im Kundensupport erzielt wird. Für zukünftige Verbesserungen
ist eine Erweiterung des gelabelten Datensatzes erforderlich, um ein feineres Tuning, insbesondere bei
unterrepräsentierten Klassen, zu ermöglichen. Hierbei kann die implementierte Human-Feedback-Loop
unterstützend wirken. Zudem sollte das Few-Shot-Learning-Modell erneut auf den vorgefilterten, sauberen
Daten trainiert werden, um deren Qualität weiter zu optimieren. Des Weiteren empfiehlt sich eine Ergänzung
der Klassen mit ergänzenden Antwortsnippets, um potenziell fehlende Antwortmöglichkeiten zu adressieren.
In Zukunft muss die bereitgestellte Schnittstelle mit dem ERP-System von Allbranded verbunden werden, um die
Vorhersagen in der Praxis anwenden zu können. 