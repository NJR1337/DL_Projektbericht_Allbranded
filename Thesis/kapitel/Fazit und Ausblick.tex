\chapter{Fazit und Ausblick}

Der Einsatz der Modelle über die entwickelte Schnittstelle erweist sich als praktikabel. Das Sentiment wird mit dem trainierten Modell zuverlässig erkannt, während Intents, je nach Klassencharakteristik, unterschiedlich gut identifiziert werden. Insbesondere häufig vorkommende Anfragetypen können effektiv klassifiziert werden, wodurch eine Zeitersparnis im Kundensupport erzielt wird. Das Klassifizieren des Sentiments ermöglicht die Erhebung einer Kundenzufriedenheitsmetrik bei Allbranded.
Für zukünftige Verbesserungen ist eine Erweiterung des gelabelten Datensatzes, insbesondere bei unterrepräsentierten Klassen, erforderlich. Hierbei kann die implementierte Human-Feedback-Loop unterstützend wirken. Zudem sollte das Few-Shot-Learning-Modell erneut auf den vorgefilterten, sauberen Daten trainiert werden, um deren Qualität weiter zu optimieren. Des Weiteren empfiehlt sich eine Ergänzung der Klassen mit ergänzenden Antwortsnippets, um potenziell fehlende Antwortmöglichkeiten zu adressieren. In Zukunft muss die bereitgestellte Schnittstelle mit dem ERP-System von Allbranded verbunden werden, um die Vorhersagen in der Praxis anwenden zu können. Das bereitgestellte User Interface kann genutzt werden, den Mitarbeitern im Kundensupport die Nutzbarkeit nahe zu bringen. 