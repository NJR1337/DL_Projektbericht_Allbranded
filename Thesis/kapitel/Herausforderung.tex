\chapter{Herausforderung}

Im Verlauf unseres Projektes sind verschiedene Herausforderungen aufgetreten, die sich auf unser
Projektergebnis ausgewirkt haben. Ein zentrales Problem stellte dabei die späte Bereitstellung sowohl
der benötigten Server-Infrastruktur als auch der Trainingsdaten durch unser Partnerunternehmen dar.
Aufgrund datenschutzrechtlicher Bedenken, die insbesondere darauf zurückzuführen waren, dass es sich
bei den Trainingsdaten um echte Kundenmails handelte, kam es zu Verzögerungen. Dies führte dazu, dass
einige Folgeprobleme erst im späten Projektverlauf auftraten, was wiederum die Möglichkeiten einer
zeitnahen und effektiven Problemlösung erheblich einschränkte. 

Wie schon in vorherigen Kapiteln erwähnt mussten wir die Trainingsdaten selbst labeln, welches einen
erheblichen Zusatzaufwand darstellte. Ebenfalls hatte das fehlende Verständnis über die veschiedenen
Prozesse bei Allbranded einen Einfluss auf die Qualität der Labels. Eine Intensivere Abstimmung mit
unserem Partnerunternehmen hätte die Qualität der Labels erhöhen können, diese war aber leider nur
zum Teil möglich, da unsere Ansprechpartner selber sehr stark ausgelastet waren.  

Darüber hinaus stellte die unbalancierte Verteilung der Kategorien innerhalb des Datensatzes eine
Schwierigkeit dar, welche die Aussagekraft und Performance der Modelle negativ beeinflusste. Verstärkt
wurde dieses Problem zusätzlich durch die generell geringe Menge an verfügbaren Daten verstärkt.  

Trotz der genannten Herausforderungen erzielte das Projekt gute Ergebnisse, welche durch die im folgenden
Kapitel dargestellten Punkte weiter verbessert werden können. 

Partnerunternehmen oder Allbranded  